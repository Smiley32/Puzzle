\documentclass[]{article}

\usepackage[utf8]{inputenc}
\usepackage[T1]{fontenc}
\usepackage[french]{babel}

\title{Algo-Prog\\Sujet \no2 : Puzzle}
\author{Jeannin Émile, Mottet Théo}

\begin{document}
\maketitle

\section{Mode d'emploi}

Le but du jeu est de reconstituer l'image dans la grille se trouvant à gauche de l'écran. Pour cela, le joueur peut cliquer sur une pièce, ce qui aura pour conséquence de voir la pièce suivre le curseur. Il faudra recliquer pour déposer la pièce. Si le curseur se trouve au dessus d'une des cases de la grille, l'image tenue sera déposée correctement dans la case (cela compte comme un coup).

Pour que le puzzle soit terminé, il faut que chaque pièce soit dans la bonne case et qu'elle ne soit pas tournée (la rotation de la pièce doit être de 0 degrés).

L'utilisateur peut tourner une piece de 90 degrés dans le sens des éguilles d'une montre en effectuant un clic droit sur une pièce. Cela a compte comme un coup.

A la fin du jeu, le temps est affiché ainsi que le nombre de coups que l'utilisateur a joué afin de résoudre le puzzle. On peut alors cliquer quelque part pour essayer un nouveau puzzle, ou bien simplement quitter l'application.

Enfin, plusieurs boutons sont à la disposition du joueur : pour changer de puzzle (aléatoirement), pour changer le nombre de pièces (a aussi pour effet de changer de puzzle) et pour activer/désactiver la rotation des pièces.

\section{Résolution des problèmes}

Problèmes (à mettre dans l'ordre):
\begin{itemize}
	\item
		Affichage (Tracage des pièces à l'écran, transparence, tjrs même fonction affichage, pzl passé en param);
	\item
		Placement des pièces dans la grille (Quand on clique au dessus d'une case en tenant une pièce, comment cell-ci est-elle positionnée dans la grille --> Est-elle au bon endroit ou non.);
	\item
		Condition de fin (Toutes les pièces placées correctement dans la grille, bonne rotation --> comment le sait-on?);
	\item
		Déplacement des pièces (Comment la pièce suit le curseur et est devant les autres, au premier clic, on tient l'image, au second on la lache);
	\item
		Rotation des pièces (Comment se passe la rotation : formule ; au clic droit, on lache l'image si on la tenait et on la fait tourner de 90°);
	 \item
		Forme des pièces (Comment on fait les pics sur chaque côtés des images --> les pics sont complémentaires entre les images ; schéma pour comprendre);
	\item
		Clic (Systeme de gestion des clics pour simplifier leur utilisation --> Comment et pourquoi);
	\item
		Découpage de l'image (Comment on découpe l'image et pourquoi lors de l'initialisation);
	\item
		Boutons (Mise en place d'un systeme de création et d'affichage de boutons : comment il marche et pourquoi l'avoir fait);
	\item
		Nouvelle partie (Mise en place d'un système permettant de rejouer --> Comment ça marche);
	\item
		Bordures des pièces (On dessine un trait épais noir sur le bord extérieur du puzzle --> comment ça marche);
	\item
		Nombre de coups et temps (Les coups sont comptés et le temps aussi).
\end{itemize}

\end{document}