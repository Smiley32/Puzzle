\documentclass[]{article}

\usepackage[utf8]{inputenc}
\usepackage[T1]{fontenc}
\usepackage[french]{babel}

\title{Algo-Prog\\Sujet \no2 : Puzzle}
\author{Jeannin Émile, Mottet Théo}

\begin{document}
\maketitle

\section{Mode d'emploi}

Le but du jeu est de reconstituer l'image dans la grille se trouvant à gauche de l'écran. Pour cela, le joueur peut cliquer sur une pièce, ce qui aura pour conséquence de voir la pièce suivre le curseur. Il faudra recliquer pour déposer la pièce. Si le curseur se trouve au dessus d'une des cases de la grille, l'image tenue sera déposée correctement dans la case (cela compte comme un coup).

Pour que le puzzle soit terminé, il faut que chaque pièce soit dans la bonne case et qu'elle ne soit pas tournée (la rotation de la pièce doit être de 0 degrés).

L'utilisateur peut tourner une piece de 90 degrés dans le sens des éguilles d'une montre en effectuant un clic droit sur une pièce. Cela a compte comme un coup.

A la fin du jeu, le temps est affiché ainsi que le nombre de coups que l'utilisateur a joué afin de résoudre le puzzle. On peut alors cliquer quelque part pour essayer un nouveau puzzle, ou bien simplement quitter l'application.

Enfin, plusieurs boutons sont à la disposition du joueur : pour changer de puzzle (aléatoirement), pour changer le nombre de pièces (a aussi pour effet de changer de puzzle) et pour activer/désactiver la rotation des pièces.

\section{Résolution des problèmes}

Je sé pa koi maître...


\end{document}